\presub %\vvv
\subsection{Evaluation on Finding \taskthree} %\postsub
\label{eva_three}

\noindent\textbf{Parameter Setting:}
%We compare three frameworks: \sketchname, \EHname and \Splittername. For each frameworks, we using CM Sketch, CM-CU Sketch and Count Sketch approaches.
%
%We compare 5 approaches: CM \sketchname, CM-CU \sketchname, Count \sketchname, \EHname {} and \Splittername.
%
We compare 3 algorithms: \sketchname, \perpie\cite{persisitem}, and \perss\cite{smallspace}.
Let $z$ be the number of hash functions for the Bloom filter. For \sketchname{}, we set $z=3$.
For \perpie{} and \perss, the parameters are set according to the recommendation of the authors.
In the experiment, we compare AAE, ARE, PR, CR, and insertion speed among the 3 algorithms.
The memory size ranges from 16MB to 19MB. We choose this range because \perpie{} cannot report any interesting items on the synthetic dataset if memory size is less than 16MB. Also, we set the memory size of \sketchname{} to $1/20$ of the memory size of the other algorithms when we compare AAE, ARE, PR, and CR. 
We do this because when memory size is more than 16MB, the AAE and ARE of \sketchname{} are 0, and the CR and PR of \sketchname{} are 1. To compare these algorithms more conveniently, we reduce the memory size of \sketchname.
			
			
\noindent\textbf{ARE (Figure~\ref{per_are_syn}-\ref{per_are_net}):}
We find that, on three real-world datasets, the ARE of \sketchname{} is around 771 times and 50212 times lower than \perss{} and \perpie. On the synthetic dataset, the ARE of \sketchname{} is around 115 times and 1655 times lower than \perss{} and \perpie.  


\noindent\textbf{PR (Figure~\ref{per_pr_syn}-\ref{per_pr_net}):}
We find that, on three real-world datasets, the PR of \sketchname{} is around 1.01 times and 1.23 times higher than \perss{} and \perpie. On the synthetic dataset, the PR of \sketchname{} is around 1.08 times and 6 times higher than \perss{} and \perpie.  
			
			
\noindent\textbf{Speed (Figure~\ref{per_speed_syn}-\ref{per_speed_net}):}
We find that the insertion speed of \sketchname{} is around 1.4 times and 4 times faster than \perss{} and \perpie{} on three real-world datasets and one synthetic dataset.

\noindent\textbf{AAE (Figure~\ref{per_aae_syn}-\ref{per_aae_net}) in Appendix \ref{app:fig}:}
We find that, on three real-world datasets, the AAE of \sketchname{} is around 698 times and 53543 times lower than \perss{} and \perpie. On the synthetic dataset, the AAE of \sketchname{} is around 167 times and 6095 times lower than \perss{} and \perpie. 

\noindent\textbf{CR (Figure~\ref{per_cr_syn}-\ref{per_cr_net}) in Appendix \ref{app:fig}:}
We find that, on three real-world datasets, the CR of \sketchname{} is around 1.03 times and 33 times higher than \perss{} and \perpie. On the synthetic dataset, the CR of \sketchname{} is around 1.05 times and 33 times higher than \perss{} and \perpie.  

\noindent\textbf{Summary:}
%
1) Although the memory size of \sketchname{} is only $1/20$ of the other algorithms, it still performs much better. The ARE of \sketchname{} is lower than 0.005 when its memory size is more than 800KB on three real-world datasets and one synthetic dataset.

2) The PR and CR of \sketchname{} are often more than 0.99 when its memory size is more than 800KB on three real-world datasets and one synthetic dataset. The PR and CR of \perss{} are often more than 0.95 when its memory size is more than 16MB on three real-world datasets and one synthetic dataset. However, the CR of \perpie{} is often lower than 0.2 because it wastes much of the space to record the items in every period.

3) \sketchname{} can achieve higher insertion speed compared to \perss{} and \perpie. Also, \sketchname{} will not be slower when its memory use increases. In contrast, \perss{} slows down as memory use increases.