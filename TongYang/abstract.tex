\begin{abstract}
	In high-speed data streams, a small fraction of items that have specific characteristics are often the focus, such as frequent items, heavy changes,
    %those source IP addresses with the most destination IP addresses (often known as Super-Spreaders), 
    Super-Spreaders, or persistent items. We call them interesting items.
	Most existing algorithms are designed for only one specific characteristic/interest, and use different data structures for different interests.
%	No previous work has so far considered the four characteristics above together. 
	%Another issue with existing algorithms is that, when memory space is limited, they can hardly guarantee both a high accuracy and a high speed at the same time. 
%Existing algorithms can be divided into two kinds. The first kind records all information of the stream, and thus is not memory efficient.
%The second kind manages to only record the interesting items, but it is challenging to differentiate interesting items from others.
	This paper generalizes these characteristics into one aspect, which we call finding interesting items in data streams. 
	%
	To find interesting items, we propose a generic framework, InterestSketch. 
	%
InterestSketch manages to record only interesting items.
To address the challenge of differentiating interesting items from others, we propose a key technique called Probabilistic Replacement and then Increment (PRI).
	%PRI is designed to record only interesting items with high accuracy and high speed, using limited memory space. 
	%The key idea of PRI is as follows. To insert a new item, we \textit{replace the current smallest item in the min-heap with a dynamic probability $\mathcal{P}$}. \textit{If the replacement is successful, the smallest interest (\eg, frequency) is incremented.}
%
	%Otherwise, we do not change the smallest item, but increase the probability $\mathcal{P}$.
	%
	Our theoretical proofs show that when replacement is successful, with high probability, the new item has a higher interest than the current smallest interest. 
	We conduct extensive experiments on three real datasets and one synthetic dataset, on four definitions of interesting items.
	Our experimental results show that compared with the state-of-the-art, for each definition of interest, our algorithm increases the insertion speed $2.2\sim 7.7$ times and decreases the error $74\sim 3207$ times.
	%
	All related code and datasets are open-source and available at Github anonymously \cite{opensource}.
\end{abstract}